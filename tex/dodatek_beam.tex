\chapter{Lista instrukcji maszyny wirtualnej BEAM}
\label{cha:operacjeBeam}
%---------------------------------------------------------------------------

Dodatek zawiera listę instrukcji maszyny wirtualnej BEAM, jakie może zawierać skompilowany kod pośredni przez nią wykonywany.
Lista zawiera nazwę operacji, jej argumenty oraz opis jej działania.

Kod operacji oraz każdy argument zajmują zawsze 1 bajt w pliku skompilowanego kodu pośredniego.
Kolejość bajtów w zapisie kodu pośredniego to \emph{big endian}.


\section{Typy argumentów}
\label{sec:opsTypes}
%---------------------------------------------------------------------------


\section{Lista instrukcji}
\label{sec:opsOps}
%---------------------------------------------------------------------------

\begin{longtable}{|c|c|c|p{5cm}|}
\hline

\multicolumn{2}{|c|}{\textbf{Kod operacji}} & \multirow{2}{*}{\textbf{Nazwa operacji i jej argumenty}} & \multirow{2}{*}{\textbf{Opis operacji i uwagi}} \\
\cline{1-2}
\textbf{szesnastkowo} & \textbf{dziesiętnie} & & \\
\hline
\endfirsthead

01 & 1 & label Lbl & Wprowadza lokalną dla danego modułu etykietę identyfikującą aktualne miejsce w kodzie. \\
\hline
02 & 2 & func\_info M F A & Definiuje funkcję F, w module M o arności A. \\
\hline
03 & 3 & int\_code\_end & ???  \\
\hline

\caption{Lista operacji maszyny wirtualnej BEAM}  \\
\end{longtable}



















