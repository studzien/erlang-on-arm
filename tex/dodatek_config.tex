\chapter{Konfiguracja parametrów maszyny wirtualnej}
\label{cha:config}
%---------------------------------------------------------------------------

Dodatek opisuje parametry konfiguracyjne maszyny wirtualnej Erlanga zaimplementowanej w ramach pracy.

Konfiguracji dokonuje się w pliku źródłowym \texttt{config.h} przed kompilacją kodu maszyny wraz z~kodem pośrednim.

Tabela \ref{table:config} przedstawia możliwe do skonfigurowania wartości razem z wartościami domyślnymi oraz opisem ich znaczenia.

\begin{longtable}{|c|c|p{8cm}|}
\hline

Parameter & Wartość domyślna & Opis \\
\endfirsthead
\hline
\texttt{THREADED\_CODE} & \texttt{1} & Jeżeli wartość parametru ustawiona jest na \texttt{0} to interpretacja kodu pośredniego odbywa się za pomocą \emph{Switch Threading}. Efektem każdego innego ustawienia jest użycie podejścia \emph{Direct Threading}. \\
\hline
\texttt{REDUCTIONS} & \texttt{300} & Liczba redukcji procesu, po wykonaniu których proces zostaje wywłaszczony przez planistę.\\
\hline
\texttt{MAX\_PROCESSES} & \texttt{4} & Liczba procesów, dla których zostanie zaalokowane miejsce w pamięci w~momencie uruchomienia maszyny. Równoznaczne z maksymalną liczbą procesów, jakie mogą być uruchomione w systemie.\\ 
\hline
\texttt{ATOM\_TABLE\_SIZE} & \texttt{100} & Liczba atomów, dla których zostanie zaalokowane miejsce w pamięci w momencie uruchomienia maszyny. Równoznaczne z maksymalną liczbą atomów, jakie mogą zostać dodane do tablicy atomów (włącznie z atomami zdefiniowanymi przez samą maszynę wirtualną).\\
\hline
\texttt{EXPORT\_TABLE\_SIZE} & \texttt{100} & Liczba adresów do funkcji zewnętrzny, dla których zostanie zaalokowane miejsce w pamięci w momencie uruchomienia maszyny. Równoznaczne z maksymalną liczbą funkcji, które mogą zostać wyeksportowane z modułów (włącznie z funkcjami wbudowanymi zdefiniowanymi przez samą maszynę wirtualną).\\
\hline
\texttt{CODE\_BUFFER\_SIZE} & \texttt{1800} & Liczba bajtów, jakich tymczasowo używa moduł ładujący kod do załadowania pojedynczego modułu do pamięci. Próba załadowania modułu, którego kod przekracza ten rozmiar zakończy się niepowodzeniem.\\
\hline
\texttt{MAX\_REG} & \texttt{255} & Liczba rejestrów \textbf{X}, jakie zostają zaalokowane dla interpretera kodu pośredniego.\\
\hline
\texttt{DEBUG} & \texttt{1} & Każda wartość inna niż \texttt{0} uaktywnia tryp deweloperski, zawierający funkcje wypisujące w konsoli wyrażenia pomocnicze przy rozwoju aplikacji.\\
\hline
\texttt{DEBUG\_OP} & \texttt{0} & Każda wartość inna niż \texttt{0} uaktywania tryb deweloperski maszyny wirtualnej, wypisujący m.in. aktualnie wykonywane instrukcje. Użycie trybu drastycznie spowalnia działanie całego systemu.\\
\hline
\texttt{TASK\_STACK\_SIZE} & \texttt{350} & Określa rozmiar stosu alokowane dla zadania systemu FreeRTOS w słowach maszynowych. \\
\hline

\caption{Parametry konfiguracyjne maszyny wirtualnej} 
\label{table:config} \\
\end{longtable}